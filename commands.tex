% electromagnetic constants
\newcommand{\numeps}{8.85 \times 10^{-12} \, \unit{C^2N^-1m^{-2}}}


\newcommand{\eps}{\varepsilon}
\renewcommand{\vec}[1]{\mathbf{#1}}
%\renewcommand{\(}{\left(}
%\renewcommand{\)}{\right)}
%\renewcommand{\d}{\cdot}
\newcommand{\Lr}{\Leftrightarrow}
\newcommand{\eq}{\begin{eqnarray}}
\newcommand{\eqs}{\begin{eqnarray*}}
\newcommand{\eqe}{\end{eqnarray}}
\newcommand{\eqes}{\end{eqnarray*}}

%faster typing partials

\newcommand{\ddx}{\frac{\partial}{\partial x}}
\newcommand{\ddy}{\frac{\partial}{\partial y}}
\newcommand{\ddz}{\frac{\partial}{\partial z}}

\newcommand{\pdiff}[2]{\frac{\partial #1}{\partial #2}}
\newcommand{\ppdiff}[3]{\frac{\partial^2 #1}{\partial #2 \partial #3}}
\newcommand{\diff}[2]{\frac{d #1}{d #2}}

\newcommand{\A}{\vec{A}}
\newcommand{\B}{\vec{B}}
\newcommand{\C}{\vec{C}}
\newcommand{\F}{\vec{F}}
\newcommand{\E}{\vec{E}}

%\renewcommand{\a}{\vec a}
%\renewcommand{\b}{\vec b}
%\renewcommand{\c}{\vec c}
%\renewcommand{\v}{\vec v}
\newcommand{\bmu}{\mbox{\boldmath{$\mathbf{\mu}$}}}

%\renewcommand{\i}{\mathbf{\hat i}}
%\renewcommand{\j}{\mathbf{\hat j}}
%\renewcommand{\k}{\mathbf{\hat k}}
%\renewcommand{\l}{\vec l}
%\renewcommand{\r}{\mathbf{\hat r}}


\newcommand{\emf}{\mathcal{E}}

%command for bold nabla
\newcommand{\bnabla}{\mbox{\boldmath$\nabla$}} 
\newcommand{\del}{\bnabla}
\newcommand{\btheta}{\mbox{\boldmath{$\mathbf{\hat{\theta}}$}}}

\newcommand{\bphi}{\mbox{\boldmath{$\mathbf{\hat{\phi}}$}}}
\newcommand{\utheta}{\mbox{\boldmath{$\mathbf{\hat{\theta}}$}}}
\newcommand{\uphi}{\mbox{\boldmath{$\mathbf{\hat{\phi}}$}}}

\newcommand{\btau}{\mbox{\boldmath{$\mathbf{\tau}$}}}

\newcommand{\x}{\mathbf{\hat x}}
\newcommand{\y}{\mathbf{\hat y}}
\newcommand{\z}{\mathbf{\hat z}}
\newcommand{\n}{\mathbf{\hat n}}

\newcommand{\p}{\vec p}

\newcommand{\ra}{\mathbf{\hat r}}


%\newcommand{\Ohm}{\Omega}
%\newcommand{\ohm}{\Omega}
% Make units have space in front of them
\let\oldunit\unit
\renewcommand{\unit}[1]{\,\oldunit{#1}}

\newcommand{\Cal}[1]{\ensuremath{\mathcalligra{#1}}}
%\newcommand{\gr}{\Cal{r}} % Griffiths style r, but is italic. See fix below (\griffiths).
\newcommand{\gr}{r}
\newcommand{\griffiths}{\includegraphics[height=5pt]{../images/scriptr.pdf}}

%del operator, divergence and curl
\newcommand{\delo}{\frac{\partial}{\partial x}\i+\frac{\partial}{\partial y}\j + \frac{\partial}{\partial z}\k}

\renewcommand{\div}[3]{\frac{\partial}{\partial x} #1 +\frac{\partial }{\partial y} #2 + \frac{\partial }{\partial z} #3}

\newcommand{\adiv}[3]{\frac{\partial #1}{\partial x}  +\frac{\partial #2}{\partial y}  + \frac{\partial #3 }{\partial z} }


\newcommand{\curl}[3]{\( \frac{\partial }{\partial y} #3 - \frac{\partial }{\partial z} #2 \)\i + \( \frac{\partial}{\partial z} #1 - \frac{\partial}{\partial x} #3\) \j + \( \frac{\partial }{\partial x} #2 - \frac{\partial}{\partial y} #1\)\k}

\newcommand{\acurl}[3]{\( \frac{\partial #3 }{\partial y} - \frac{\partial #2}{\partial z}  \)\i + \( \frac{\partial  #1}{\partial z} - \frac{\partial #3}{\partial x} \) \j + \( \frac{\partial  #2}{\partial x} - \frac{\partial #1}{\partial y} \)\k}


\newcommand{\mcurl}[3]{\left|\begin{matrix} \i & \j & \k \\
  \frac{\partial}{\partial x} & \frac{\partial}{\partial y} & \frac{\partial}{\partial z}\\ #1 & #2 & #3 \end{matrix}\right|}

\newcommand{\grad}[1]{\frac{\partial}{\partial x} #1\i +\frac{\partial }{\partial y} #1 \j + \frac{\partial }{\partial z} #1\i}

\newcommand{\agrad}[1]{\frac{\partial #1}{\partial x} \i +\frac{\partial #1 }{\partial y} \j + \frac{\partial #1}{\partial z} \k}


%%%% Custom Command for floating Infoboxes
%%%% usage: \infobox{<title>}{<text>}
\definecolor{infobox}{rgb}{0.9,0.95,1.0}      
\newcommand{\infobox}[1]{
\begin{wrapfigure}{r}{0.3\textwidth}
    %\colorbox{infobox}{
    %\framebox{
        \hspace{10pt}
        \parbox{0.3\textwidth}{
    
            \begin{wrapfigure}{r}{0.05\textwidth}
                \vspace{-23pt}
                \includegraphics[width=0.05\textwidth]{../images/info.png}
                \vspace{-23pt}
            \end{wrapfigure}

            \small{\par #1\par} 
    %}
        }
\end{wrapfigure}
}

%%%% Custom command to print out constants (so we don't have to change the formatting in 400 files)
%%%% usage: \constant{<title>}{<definition}
\newcommand{\constant}[2]{#1: & #2 \\[3pt]}

%%%% Custom command to list constants
\newenvironment{constants}
{
    \section*{Useful constants} \label{sec:usefulconstants}
    
    These constants might be useful in some of this week's exercises.

    \begin{center}
    \begin{tabular}{lr}
    \hline
    \noalign{\smallskip}
}
{
    \hline
    \end{tabular}
    \end{center}
}

%%%% Custom command to print out exercises
\newcounter{week}
\newcounter{exercise}
\newcommand{\exercise}[1]{
    \setcounter{qcounter}{0}
    \refstepcounter{exercise} \section*{Exercise \arabic{week}.\arabic{exercise}: #1}
}



%%%% Custom command to print out questions
\newcounter{qcounter}
\newenvironment{questions}              % environment name
{
    \begin{enumerate}[a)]
    \setcounter{enumi}{\value{qcounter}}
}                 % begin code
{
    \setcounter{qcounter}{\value{enumi}}
    \end{enumerate}
}                       % end code
\newcommand{\question}{\item}
\newcommand{\questiono}{\item \textbf{Obligatory:} }


\usepackage{comment}

%%%% Answers
%%%% Check if \disableanswers is defined or not. If not, print answers.
\definecolor{solutioncolor}{rgb}{0.0,0.0,0.5}
\ifx \disablesolution \undefined
    \newenvironment{sol}
    {
        \par \vspace{10pt} 
        %\colorlet{shadecolor}{yellow!10}
        %\begin{shaded}
        \color{solutioncolor}
        \emph{Solution:}
    }
    {
        %\begin{center}
        %\line(1,0){420}
        %\end{center} 
        %\end{shaded}
        \vspace{10pt} \par
    }
\else
    \excludecomment{sol}
%     \newenvironment{sol}{\comment}{\endcomment}
%     \newenvironment{sol}{}{}
%     \newenvironment{sol}{\ifx \disablesolution \undefined}{\fi}
	%\newenvironment{sol}{}{}
\fi
\newcommand{\solution}[1]{\begin{sol} #1 \end{sol}}

\definecolor{answercolor}{rgb}{0.0,0.5,0.0}
\ifx \disableanswers \undefined
    \newenvironment{ans}
    {
        \par  \vspace{10pt}
        %\colorlet{shadecolor}{yellow!10}
        %\begin{shaded}
        \color{answercolor}
        \emph{Answer:}
    }
    {
        %\end{shaded}
        \vspace{15pt} \par
    }
    %\newcommand{\answer}[1]{}   % Answers visible
\else
    \excludecomment{ans}
    %\newcommand{ans}[1]{} % Answers hiddden
    %\newenvironment{ans}{}{}
\fi
\newcommand{\answer}[1]{\begin{ans} #1 \end{ans}}

\newcommand{\defaultheader}[1]{
  \lfoot{\small{Week \arabic{week} -- #1}}
  \rfoot{\tiny{compiled \today}}
  \lhead{}
  \rhead{}
  \pagestyle{fancy}
}

\newcommand{\degree}{\ensuremath{^\circ}}

\newcounter{example}
\newcommand{\example}[1]{
    \setcounter{qcounter}{1}
    \refstepcounter{example} \subsection*{Example #1}
}

\newcounter{lqcounter}
\newenvironment{labquestions}              % environment name
{
  \colorlet{shadecolor}{yellow!10}
    \begin{shaded}
    {\bf Questions:}
    \begin{enumerate}
    \setcounter{enumi}{\value{qcounter}}
}                 % begin code
{
    \setcounter{qcounter}{\value{enumi}}
    \end{enumerate}
    \end{shaded}
    
}                     % end code


% BEGIN COMMANDS FROM LATEX FOR PHYSICISTS
% http://www.dfcd.net/articles/latex/latex.html
\newcommand{\ket}[1]{\left| #1 \right>} % for Dirac bras
\newcommand{\bra}[1]{\left< #1 \right|} % for Dirac kets
\newcommand{\braket}[2]{\left< #1 \vphantom{#2} \right|
 \left. #2 \vphantom{#1} \right>} % for Dirac brackets
% END COMMANDS FROM
